% Please do not change the document class
\documentclass{scrartcl}

% Please do not change these packages
\usepackage[hidelinks]{hyperref}
\usepackage[none]{hyphenat}
\usepackage{setspace}
\doublespace

% You may add additional packages here
\usepackage{amsmath}

% Please include a clear, concise, and descriptive title
\title{Continuing Personal Development Report}

% Please do not change the subtitle
\subtitle{COMP150 - CPD Report}

% Please put your student number in the author field
\author{1801507}

\begin{document}

\maketitle

\section{Introduction}
My career goal is to become a professional game programmer and establish my own indie studio. The five key skills that I have faced the most problems with so far and would like to work on the most are: maintaining focus; designing and executing plans; being collegiate; maintaining programming practice to learn new C++ constructs; and avoiding frustration when fixing bugs.

\section{Maintaining Focus (Dispositional)}

Indie development is risky and requires fast turn-around of projects. So, focus is essential. However, I find it difficult to focus myself. I have a tendency to overwhelm myself by taking on too many tasks at once, which usually leads to me being unproductive. I especially notice this when having to work on multiple projects, such as the Python game alongside the agile essay and exercise sheets. I saw how much I needed to do and would feel apprehensive. It's only when the urgency becomes critical that I feel it overcome this apprehension. To overcome this problem, I will plan my time in order to focus on one task at a time. However, I first need to learn how to do this. So, I will read Allen's \textit{Getting Things Done} and find at least one principle I can apply. I will then experiment with this approach for the Unreal project week-by-week. At the end of each week, I will conduct a retrospective and decide whether or not to practice more or apply new principles.

\section{Designing \& Executing Plans (Procedural)}

Another problem of mine is that when I get around to planning something, or I schedule a meeting with someone, I struggle to execute these plans. I noticed that especially in the Python game project, where I would often not show up on days where we planned to meet up, talk things over or do some work together. I also notice it in my everyday life with all my failed attempts to organize myself, or start doing something new. To overcome this problem, at the end of each week I will consult my friends, show them my retrospectives as well as my weekly plans and accomplishments, in order to receive feedback and put pressure on myself. I will do that after I write the first retrospective at the end of next week.

\section{Being Collegiate (Interpersonal)}

When working on a group project, one has to be collegiate to others, which means being responsible and reliable. A good example of collegiate behaviour is e.g, showing up on time. The underlying cause of problems to this end, is my struggle to take care of health and well-being, which also resulted in less-than collegiate interaction with my team and group projects. Since coming to uni I have had quite a financial struggle and still have problems eating properly. I also spend most of my time sitting at home and don't go out much or engage in any physical activities, e.g, exercising at the gym. I also have trouble sleeping, which is partly a side effect of my unhealthy lifestyle. To overcome this problem, I first need to find a part time job and stabilize my financial situation. I will make sure to reserve some time for job hunting next week, and after I fixed my financial situation I will be able to get myself some proper food, start going to the gym etc.

\section{Learning New C++ Constructs (Cognitive)}

I feel like I waste too much time on things like watching YouTube videos or just sleeping too much. If I spent that wasted time on programming practice or learning new c++ constructs I feel like I would be much better off. I feel like I'm starting to slowly start lagging behind when it comes to Unreal and c++ and I haven't got much practice recently. This problem is also a side effect of some previously described struggles of mine. I will really need to refocus on this area in the following weeks as the Unreal game project is starting to kick-off. To help me accomplish that, I will give myself a time limit for wasteful activities everyday starting from next week and spend as much of my time as I can being productive.

\section{Fixing Bugs (Affective)}

I have wasted tons of time getting stressed and frustrated over various bugs and sitting there for hours without a break trying to solve them by blankly staring at the screen. I encountered many of these moments of frustration while working on my Python game project and when playing Space Chem. But I also managed to experience what it's like when you actually take a break, let your brain rest and sometimes get an eureka moment. I learned how important it is to give your brain a break and have a moment to relax and sort your thoughts. Now I will make sure to always remember that, when I encounter a bug and feel like I'm not getting anywhere, it's time for a moment of rest. Tonight I will write down the information from the previous sentence somewhere easily visible for me, in order to remind myself of this principle.

\section{Conclusion}

To achieve my goal of becoming a professional games programmer and someday set up an indie game development studio, I will need to overcome the problems that I specified in this report. To accomplish that I will read Allen's \textit{Getting Things Done} and experiment with applying principles from that book to my Unreal project. I will also write a retrospective at the end of each week. I will show the retrospectives and all my weekly plans and accomplishments to my friends, in order to receive feedback and put pressure on myself. I will also reserve some time for job hunting and after I found a job and stabilized my financial situation, I will make sure to take better care of my diet and physique. I will also write down a reminder to take a break when struggling with a bug in an easily visible place.

\bibliographystyle{ieeetran}
\bibliography{references}

\end{document}